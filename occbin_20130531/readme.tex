%2multibyte Version: 5.50.0.2890 CodePage: 65001

\documentclass[12pt]{article}
%%%%%%%%%%%%%%%%%%%%%%%%%%%%%%%%%%%%%%%%%%%%%%%%%%%%%%%%%%%%%%%%%%%%%%%%%%%%%%%%%%%%%%%%%%%%%%%%%%%%%%%%%%%%%%%%%%%%%%%%%%%%%%%%%%%%%%%%%%%%%%%%%%%%%%%%%%%%%%%%%%%%%%%%%%%%%%%%%%%%%%%%%%%%%%%%%%%%%%%%%%%%%%%%%%%%%%%%%%%%%%%%%%%%%%%%%%%%%%%%%%%%%%%%%%%%
\usepackage{amsfonts}
\usepackage{graphicx}
\usepackage{amsmath}
\usepackage{rotating}
\usepackage{ulem}

\setcounter{MaxMatrixCols}{10}
%TCIDATA{OutputFilter=LATEX.DLL}
%TCIDATA{Version=5.50.0.2890}
%TCIDATA{Codepage=65001}
%TCIDATA{<META NAME="SaveForMode" CONTENT="1">}
%TCIDATA{BibliographyScheme=Manual}
%TCIDATA{Created=Thu May 25 17:12:58 2000}
%TCIDATA{LastRevised=Friday, June 07, 2013 10:26:22}
%TCIDATA{<META NAME="GraphicsSave" CONTENT="32">}
%TCIDATA{<META NAME="DocumentShell" CONTENT="Journal Articles\Standard LaTeX Article">}
%TCIDATA{Language=American English}
%TCIDATA{CSTFile=LaTeX article (bright).cst}

\newtheorem{theorem}{Theorem}
\newtheorem{acknowledgement}[theorem]{Acknowledgement}
\newtheorem{algorithm}[theorem]{Algorithm}
\newtheorem{axiom}[theorem]{Axiom}
\newtheorem{case}[theorem]{Case}
\newtheorem{claim}[theorem]{Claim}
\newtheorem{conclusion}[theorem]{Conclusion}
\newtheorem{condition}[theorem]{Condition}
\newtheorem{conjecture}[theorem]{Conjecture}
\newtheorem{corollary}[theorem]{Corollary}
\newtheorem{criterion}[theorem]{Criterion}
\newtheorem{definition}[theorem]{Definition}
\newtheorem{example}[theorem]{Example}
\newtheorem{exercise}[theorem]{Exercise}
\newtheorem{lemma}[theorem]{Lemma}
\newtheorem{notation}[theorem]{Notation}
\newtheorem{problem}[theorem]{Problem}
\newtheorem{proposition}[theorem]{Proposition}
\newtheorem{remark}[theorem]{Remark}
\newtheorem{solution}[theorem]{Solution}
\newtheorem{summary}[theorem]{Summary}
\newenvironment{proof}[1][Proof]{\textbf{#1.} }{\  \rule{0.5em}{0.5em}}
\input{tcilatex}
\setlength{\textheight}{22.5cm}
\setlength{\textwidth}{16.5cm}
\setlength{\oddsidemargin}{0cm}
\setlength{\evensidemargin}{0cm}
\setlength{\topmargin}{-1cm}

\begin{document}


%TCIMACRO{\TeXButton{no_numbering_chapter}{\setcounter{secnumdepth}{-1}}}%
%BeginExpansion
\setcounter{secnumdepth}{-1}%
%EndExpansion

%TCIMACRO{%
%\TeXButton{stretch}{\renewcommand{\baselinestretch}{1.05}
%\normalsize}}%
%BeginExpansion
\renewcommand{\baselinestretch}{1.05}
\normalsize%
%EndExpansion

\begin{center}
\Large
\textbf{USER GUIDE FOR GUERRIERI-IACOVIELLO OCCBIN TOOLKIT}
\normalsize

\vspace{1cm}
JUNE 7, 2013
\end{center}

CITE AS: Guerrieri, Luca, and Matteo Iacoviello (2013) ``Occbin:\ A Toolkit
to Solve Models with Occasionally Binding Constraints Easily'', working
paper, Federal Reserve Board

\bigskip

\section{Overview}

\begin{enumerate}
\item Modify and run the file \texttt{setpathdynare4.m} so as to point to
the local Dynare installation directory and to the directory containing the 
\texttt{toolkit\_files}.

We have used successfully Dynare versions 4.3.1 on Windows and 4.3.3 on Mac.

\item LIST\ OF\ THE\ EXAMPLE\ FILES (in increasing order of complexity and
decreasing order of documentation)

\begin{enumerate}
\item \texttt{runsim\_irrcap.m.} An RBC\ model with a constraint on the
level of investment. The two relevant mod files are dynrbc.mod and
dynrbcirr\_i.mod.

This examples shows how to declare the parameter values in an extermal file
(named paramfile\_irrcap.m, called from dynrbc\_steadystate.m).

If the example file runs correctly, it will generate a figure like the one
in the file figure\_example.pdf

\item \texttt{runsim\_borrcon.m} \ In this example, \texttt{borrcon.mod} is
a model with borrowing constraint that binds occasionally.

\item \texttt{runsim\_irrcap\_twoconstraints.m. }An RBC\ model with a
constraint on the level of investment and an asymmetric capital adjustment
cost. This file shows how one can use the codes to solve a model with two
occasionally binding constraints using the function \texttt{%
solve\_two\_constraints}

\texttt{runsim\_irrcap\_twoconstraints\_computepolicy.m. }For the same model
as above, but uses the code and the initcon option to show one can use the
toolkit to derive the model's nonlinear policy functions.

\item \texttt{runsim\_cgg.m}. Solves a version of the Clarida-Gali-Gertler
(\textquotedblleft The Science of Monetary Policy: A New Keynesian
Perspective, JEL, 2009) model allowing for an inertial Taylor rule subject
the zero lower bound on nominal interest rates (see equation for \texttt{rnot%
}, the notional interest rate).

\item \texttt{runsim\_smetswouters.m}. Solves the Smets-Wouters (AER, 2007)
model allowing for the zero lower bound on nominal interest rates (see
equation for \texttt{rnot}, the notional interest rate). The codes for the
model were downloaded from the online Appendix on the AEA webpage. We
affected minimal changes to the code for compatibility with Dynare 4. To
avoid the estimation step, some parameter values were set at their initial
level prior to estimation.

\item \texttt{runsim\_dnk.m. }Solve a new-keynesian model with zero lower
bound and government spending. This folder shows how one can use the codes
to declare the parameter values only once in an outside file (named
paramfile\_dnk.m). Shows how one can use separate sets of functions to solve
model disregarding nonlinearities, or to compute impulse responses
conditional on different baseline paths for the variables.

-- \texttt{dnk.mod} contains a standard new-keynesian model specified away
from the zlb constraint.

-- \texttt{dnk\_zlb.mod} is an exact replica of dnk.mod file with the model
specified at the constraint.

Except for the interest rate equation, the models in the two .mod files are
identical.
\end{enumerate}

\item The only restrictions for the .mod files is that they may accommodate
at most one lag and one lead of the endogenous variables. The conditions
that govern the switching across the reference and alternative(s) model(s)
may only involve contemporaneous variables. With appropriate redefinitions,
these restrictions come at no loss of generality.

\item The function that solves the model is:

\texttt{solve\_one\_constraint}

\texttt{[zdatalinear zdatapiecewise zdatass oo\_base M\_base] = }

\texttt{solve\_one\_constraint(modnam, modnamstar, }

\texttt{constraint, constraint\_relax, }

\texttt{shockssequence, irfshock, nperiods, maxiter, init);}

\medskip 

\underline{Inputs:}

\texttt{modnam}: name of .mod file for the reference regime (excludes the
.mod extension).

\texttt{modnamstar}: name of .mod file for the alternative regime (excludes
the .mod extension).

\texttt{constraint}: the constraint (see notes 1 and 2 below). When the
condition in \texttt{constraint} evaluates to \emph{true}, the solution
switches from the reference to the alternative regime.

\texttt{constraint\_relax}: when the condition in \texttt{constraint\_relax}
evaluates to \emph{true}, the solution returns to the reference regime.

\texttt{shockssequence}: a sequence of unforeseen shocks under which one
wants to solve the model (size T$\times $nshocks).

\texttt{irfshock}: label for innovation for IRFs, from Dynare .mod file (one
or more of the `varexo').

\texttt{nperiods}: simulation horizon (can be longer than the sequence of
shocks defined in \texttt{shockssequence}; must be long enough to ensure
convergence back to the reference model at the end of the simulation horizon
and may need to be varied depending on the sequence of shocks).

\texttt{maxiter}: maximum number of iterations allowed for the solution
algorithm (20 if not specified).

\texttt{init: } the initial position for the vector of state variables, in
deviation from steady state (if not specified, the default is steady state).
The ordering follows the definition order in the .mod files.

\medskip 


\underline{Outputs:}

\texttt{zdatalinear:\ } an array containing paths for all endogenous
variables ignoring the occasionally binding constraint (the linear
solution), in deviation from steady state. Each column is a variable, the
order is the definition order in the .mod files.

\texttt{zdatapiecewise:\ } an array containing paths for all endogenous
variables satisfying the occasionally binding constraint (the
occbin/piecewise solution), in deviation from steady state. Each column is a
variable, the order is the definition order in the .mod files.

\texttt{zdatass:\ } the initial position for the vector of state variables,
in deviation from steady state (if not specified, the default is a vector of
zero implying that the initial conditions coincide with the steady state).
The ordering follows the definition order in the .mod files.

\texttt{oobase\_, Mbase\_ }: \ structures produced by Dynare for the
reference model -- see Dynare User Guide.

\medskip \medskip 

\item The function that solves the model with two constraints is

\texttt{[zdatalinear zdatapiecewise zdatass oo\_00 M\_00] = }

\texttt{\
solve\_two\_constraints(modnam\_00,modnam\_10,modnam\_01,modnam\_11,...}

\texttt{\ constraint1, constraint2,...}

\texttt{\ constraint\_relax1, constraint\_relax2,...}

\texttt{\ shockssequence,irfshock,nperiods,curb\_retrench,maxiter,init);}

\medskip 

\underline{Inputs:}

\texttt{modnam\_00}: name of the .mod file for reference regime (excludes
the .mod extension).

\texttt{modnam\_10}: name of the .mod file for the alternative regime
governed by the first constraint.

\texttt{modnam\_01}: name of the .mod file for the alternative regime
governed by the second constraint.

\texttt{modnam\_11}: name of the .mod file for the case in which both
constraints force a switch to their alternative regimes.

\texttt{constraint1}: the first constraint (see notes 1 and 2 below). If 
\texttt{constraint1} evaluates to \emph{true}, then the solution switches to
the alternative regime for condition 1. In that case, if \texttt{constraint2}
(described below) evaluates to \emph{false}, then the model solution
switches to enforcing the conditions for an equilibrium in \texttt{modnam\_10%
}. Otherwise, if \texttt{constraint2} also evaluates to \emph{true}, then
the model solution switches to enforcing the conditions for an equilibrium
in \texttt{modnam\_11}.

\texttt{constraint\_relax1}: when the condition in \texttt{constraint\_relax1%
} evaluates to \emph{true}, the solution returns to the reference regime for 
\texttt{constraint1}.

\texttt{constraint2}: the second constraint (see notes 1 and 2 below).

\texttt{constraint\_relax2}: when the condition in \texttt{constraint\_relax2%
} evaluates to \emph{true}, the solution returns to the reference regime for 
\texttt{constraint2}.

\texttt{shockssequence}: a sequence of unforeseen shocks under which one
wants to solve the model

\texttt{irfshock}: label for innovation for IRFs, from Dynare .mod file (one
or more of the `varexo')

\texttt{nperiods}: simulation horizon (can be longer than the sequence of
shocks defined in \texttt{shockssequence}; must be long enough to ensure
convergence back to the reference model at the end of the simulation horizon
and may need to be varied depending on the sequence of shocks).

\texttt{curb\_retrench: } a scalar equal to 0 or 1. Default is 0. When set
to 0, it updates the guess based of regimes based on the previous iteration.
When set to 1, it updates in a manner similar to a Gauss-Jacobi scheme,
slowing the iterations down by updating the guess of regimes only one period
at a time.

\texttt{maxiter}: maximum number of iterations allowed for the solution
algorithm (20 if not specified).

\texttt{init: } the initial position for the vector of state variables, in
deviation from steady state (if not specified, the default is a vector of
zero implying that the initial conditions coincide with the steady state).
The ordering follows the definition order in the .mod files.

\underline{Outputs:}

\texttt{zdatalinear:\ } an array containing paths for all endogenous
variables ignoring the occasionally binding constraint (the linear
solution), in deviation from steady state. Each column is a variable, the
order is the definition order in the .mod files.

\texttt{zdatapiecewise:\ } an array containing paths for all endogenous
variables satisfying the occasionally binding constraint (the
occbin/piecewise solution), in deviation from steady state. Each column is a
variable, the order is the definition order in the .mod files.

\texttt{zdatass:\ } a vector that holds the steady state values of the
endogenous variables ( following the definition order in the .mod file).

\texttt{oo00\_, M00\_ }: \ structures produced by Dynare for the reference
model -- see Dynare User Guide.

\item The functions \texttt{solve\_one\_constraint} and \texttt{%
solve\_two\_constraints} assume that the endogenous, exogenous variables and
the model parameters are declared in exactly the same order in all .mod
files.

In general, to create the additional files, it pays to simply make a replica
of the reference .mod file and amend the relevant equations.

\medskip 
\end{enumerate}

\section{Additional Notes}

\begin{enumerate}
\item Suppose the original occasionally binding constraint in the model is 
\begin{equation*}
i_{t}\geq \log \left( \phi \cdot I_{ss}\right)
\end{equation*}%
where $i_{t}$ is natural logarithm of investment (see mod file), $I_{ss}$ is
steady state investment in levels, and $\phi $ is a parameter.

For the constraint to be violated, in the candidate solution calculated
under the assumption that the constraint does not bind the following must be
true%
\begin{equation*}
i_{t}<\log \left( \phi \cdot I_{ss}\right)
\end{equation*}%
Rewrite each variable as%
\begin{equation*}
x_{t}\equiv \widetilde{x}_{t}+x_{ss}
\end{equation*}

In the runsim\_irrcap.m code, the constraint will have to be expressed in
linearized form as%
\begin{equation*}
\widetilde{i}_{t}+i_{ss}<\log \left( \phi \cdot I_{ss}\right)
\Longleftrightarrow \widetilde{i}_{t}<\log \left( \phi \right)
\end{equation*}

and the string will be 
\begin{equation*}
i<log(PHII)
\end{equation*}

Therefore note that in the mod file, $i$ will denote the variable, whereas
in the \texttt{constraint} the same $i$ will denote the variable minus its
steady state level.

\item In the toolbox, \texttt{constraint} and \texttt{constraint\_relax}
only admit contemporaneous endogenous variables. Note that there is no loss
of generality since appropriate redefinitions can accommodate a general lead
and lag structure.

\item In all runsim\_*.m files, we declare M\_ and oo\_ to be global
variables (for use by Dynare)

\item One does not need to specify the steady state of any alternative
model. All models are approximated around the steady state of the reference
model (which needs to satisfy the Blanchard-Kahn conditions). Since local
and global stability need not coincide, the conditions for an equilibrium in
the alternative regime(s) need not satisfy the BK conditions.

\item Values for the parameters (whether in the .mod or in external files)
are specified only for the reference model. Parameter values specified for
the alternative model(s) are ignored by the code (but not the list of
parameters, which is used for an error check). If a parameter only enters
the alternative model, it needs to be declared as parameter and assigned a
value in the reference model.

\item We have strived to minimize possible conflicts between local variables used in coding the solution algorithm and parameter names declared in the .mod files.  We have reserved variable names with a trailing underscore -- i.e., names such as ``$myfavoritevariablename\_$'' -- 
for the solution routines. Variable names with trailing underscores should be avoided in the .mod files.

\end{enumerate}

\end{document}
