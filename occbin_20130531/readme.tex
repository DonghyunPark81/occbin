%2multibyte Version: 5.50.0.2890 CodePage: 65001

\documentclass[12pt]{article}
%%%%%%%%%%%%%%%%%%%%%%%%%%%%%%%%%%%%%%%%%%%%%%%%%%%%%%%%%%%%%%%%%%%%%%%%%%%%%%%%%%%%%%%%%%%%%%%%%%%%%%%%%%%%%%%%%%%%%%%%%%%%%%%%%%%%%%%%%%%%%%%%%%%%%%%%%%%%%%%%%%%%%%%%%%%%%%%%%%%%%%%%%%%%%%%%%%%%%%%%%%%%%%%%%%%%%%%%%%%%%%%%%%%%%%%%%%%%%%%%%%%%%%%%%%%%
\usepackage{amsfonts}
\usepackage{graphicx}
\usepackage{amsmath}
\usepackage{rotating}
\usepackage{ulem}

\setcounter{MaxMatrixCols}{10}
%TCIDATA{OutputFilter=LATEX.DLL}
%TCIDATA{Version=5.50.0.2890}
%TCIDATA{Codepage=65001}
%TCIDATA{<META NAME="SaveForMode" CONTENT="1">}
%TCIDATA{BibliographyScheme=Manual}
%TCIDATA{Created=Thu May 25 17:12:58 2000}
%TCIDATA{LastRevised=Tuesday, June 04, 2013 11:40:56}
%TCIDATA{<META NAME="GraphicsSave" CONTENT="32">}
%TCIDATA{<META NAME="DocumentShell" CONTENT="Journal Articles\Standard LaTeX Article">}
%TCIDATA{Language=American English}
%TCIDATA{CSTFile=LaTeX article (bright).cst}

\newtheorem{theorem}{Theorem}
\newtheorem{acknowledgement}[theorem]{Acknowledgement}
\newtheorem{algorithm}[theorem]{Algorithm}
\newtheorem{axiom}[theorem]{Axiom}
\newtheorem{case}[theorem]{Case}
\newtheorem{claim}[theorem]{Claim}
\newtheorem{conclusion}[theorem]{Conclusion}
\newtheorem{condition}[theorem]{Condition}
\newtheorem{conjecture}[theorem]{Conjecture}
\newtheorem{corollary}[theorem]{Corollary}
\newtheorem{criterion}[theorem]{Criterion}
\newtheorem{definition}[theorem]{Definition}
\newtheorem{example}[theorem]{Example}
\newtheorem{exercise}[theorem]{Exercise}
\newtheorem{lemma}[theorem]{Lemma}
\newtheorem{notation}[theorem]{Notation}
\newtheorem{problem}[theorem]{Problem}
\newtheorem{proposition}[theorem]{Proposition}
\newtheorem{remark}[theorem]{Remark}
\newtheorem{solution}[theorem]{Solution}
\newtheorem{summary}[theorem]{Summary}
\newenvironment{proof}[1][Proof]{\textbf{#1.} }{\  \rule{0.5em}{0.5em}}
\input{tcilatex}
\setlength{\textheight}{22.5cm}
\setlength{\textwidth}{16.5cm}
\setlength{\oddsidemargin}{0cm}
\setlength{\evensidemargin}{0cm}
\setlength{\topmargin}{-1cm}

\begin{document}


%TCIMACRO{\TeXButton{no_numbering_chapter}{\setcounter{secnumdepth}{-1}}}%
%BeginExpansion
\setcounter{secnumdepth}{-1}%
%EndExpansion

%TCIMACRO{%
%\TeXButton{stretch}{\renewcommand{\baselinestretch}{1.05}
%\normalsize}}%
%BeginExpansion
\renewcommand{\baselinestretch}{1.05}
\normalsize%
%EndExpansion

\begin{center}
\textbf{README FOR FOR THE GUERRIERI-IACOVIELLO OCCBIN TOOLKIT}

JUNE 3, 2013
\end{center}

CITE AS: Guerrieri, Luca, and Matteo Iacoviello (2013) "Occbin:\ A Toolkit
to Solve Models with Occasionally Binding Constraints Easily", working
paper, Federal Reserve Board

\bigskip

\section{Overview}

\begin{enumerate}
\item Modify and run the file \texttt{setpathdynare4.m} so as to point to
the local dynare installation directory and to the folder containing the 
\texttt{toolkit\_files} directory.

We have used successfully Dynare versions 4.3.1 on Windows and 4.2.1 on Mac.

\item LIST\ OF\ THE\ EXAMPLE\ FILES (in increasing order of complexity and
decreasing order of documentation)

\begin{enumerate}
\item \texttt{runsim\_irrcap.m. }An RBC\ model with non-negativity
constraint on investment. The two relevant mod files are dynrbc.mod and
dynrbcirr\_i.mod.

This folder shows how one can use the codes to declare the parameter values
only once in an outside file (named paramfile\_irrcap.m, called from
dynrbc\_steadystate.m).

If the example file runs correctly, it will generate a figure like the one
in the file figure\_example.pdf

\item \texttt{runsim\_borrcon.m} \ In this example, \texttt{borrcon.mod} is
a model with borrowing constraint that binds occasionally.

\item \texttt{runsim\_irrcap\_twoconstraints.m. }An RBC\ model with
non-negativity constraint on investment and an asymmetric capital adjustment
cost. This file shows how one can use the codes to solve a model with two
occasionally binding constraints using the function \texttt{%
solve\_two\_constraints}

\texttt{runsim\_irrcap\_twoconstraints\_computepolicy.m. }For the same model
as above, but uses the code and the initcon option to show one can use the
toolkit to derive the model's nonlinear policy functions.

Additional examples
\end{enumerate}

\begin{itemize}
\item \texttt{runsim\_smetswouters.m}. Solves the Smets-Wouters (AER, 2007)
model allowing for the ZLB on nominal interest rates (see equation for 
\texttt{rnot}, the notional interest rate). The codes for the model were
downloaded from the online Appendix on the AEA webpage. We affected minimal
changes to the code for compatibility with Dynare 4. To avoid the estimation
step, some parameter values were set at their initial level prior to
estimation.

\item \texttt{runsim\_dnk.m. }Solve a new-keynesian model with ZLB and
government spending. This folder shows how one can use the codes to declare
the parameter values only once in an outside file (named paramfile\_dnk.m).
Shows how one can use separate sets of functions to solve model disregarding
nonlinearities, or to compute impulse responses conditional on different
baseline paths for the variables.

-- \texttt{dnk.mod} contains a standard new-keynesian model specified away
from the zlb constraint.

-- \texttt{dnk\_zlb.mod} is an exact replica of dnk.mod file with the model
specified at the constraint.

Except for the interest rate equation, the models in the two .mod files are
identical.

\item \texttt{runsim\_nakata.m. }Solve a new-keynesian model with ZLB and
Rotemberg pricing.
\end{itemize}

The only restriction for the .mod files is that they can accommodate at most
one lag and one lead of the endogenous variables, and that the constraint
only involve contemporaneous variables. With appropriate redefinitions, this
restriction comes at no loss of generality.

\item The main program assumes that the endogenous, exogenous variables, and
parameters are declared in exactly the same order in both .mod files.

In general, to create the second file, it pays to simply make a replica of
the main .mod file that disregards the constraint and amend the relevant
equations.

\medskip

\item The function that solves the model is:

\texttt{solve\_one\_constraint}

\texttt{[zdatalinear zdatapiecewise zdatass oo\_base M\_base] = }

\texttt{solve\_one\_constraint(modnam, modnamstar, }

\texttt{constraint, constraint\_relax, }

\texttt{shockssequence, irfshock, nperiods, maxiter, init);}

\medskip

Inputs:

\texttt{modnam}: name of reference model (a Dynare .mod file applicable when
the constraint does not bind)

\texttt{modnamstar}: name of alternative model (a Dynare .mod file
applicable when the constraint binds)

\texttt{constraint}: the constraint (see note 1 below).

\texttt{constraint\_relax}: the violation of the \texttt{constraint} in the
alternative model that triggers switch to reference model (often, just
constraint with sign inverted; see example files)

\texttt{shockssequence}: a sequence of unforeseen shocks under which one
wants to solve the model (size T$\times $nshocks)

\texttt{irfshock}: label for innovation for IRFs, from Dynare mod file (one
or more of the `varexo')

\texttt{nperiods}: simulation horizon (can be longer than the sequence of
shocks defined in \texttt{shockssequence}; in some cases it needs to be long
enough to ensure convergence back to the reference model at the end of the
simulation horizon)

\texttt{maxiter}: maximum number of iterations to solve model until
convergence (20 if not specified)

\texttt{init: }the initial position for the vector of state variables, in
deviation from steady state. (If not specified, the default is steady state)

\medskip

Outputs:

\texttt{zdatalinear:\ }array containing variables ignoring the occasionally
binding constraint (the linear solution), in deviation from steady state.
Each column is a variable, the order is the definition order in the mod file.

\texttt{zdatapiecewise:\ }array containing variables satisfying the
occasionally binding constraint (the occbin/piecewise solution), in
deviation from steady state.

\texttt{zdatass:\ }steady state value of the endogenous variables (a vector,
ordered as in the mod file)

\texttt{oobase\_, Mbase\_}:\ structures produced by Dynare for the reference
model.

\medskip \medskip

\item The function that solves the model with two constraints is

\texttt{\
solve\_two\_constraints(modnam\_00,modnam\_10,modnam\_01,modnam\_11,...}

\texttt{\ constraint1, constraint2,...}

\texttt{\ constraint\_relax1, constraint\_relax2,...}

\texttt{\ shockssequence,irfshock,nperiods,curb\_retrench,maxiter,init);}

\medskip

Inputs:

\texttt{modnam\_00}: reference model

\texttt{modnam\_10}: the first constraint switches to its alternative model
but second does not

\texttt{modnam\_10}: the second constraint switches to its alternative model
but first one does not

\texttt{modnam\_11}: both constraints switch to their alternative models

\texttt{constraint1}: the first constraint (see note 1 below).

\texttt{constraint\_relax1}: the violation of \texttt{constraint1} in the
alternative model that triggers switch to reference model (often, just
constraint with sign inverted; see example files)

\texttt{constraint2}: the second constraint (see note 1 below).

\texttt{constraint\_relax2}: the violation of \texttt{constraint2} in the
alternative model that triggers switch to reference model (often, just
constraint with sign inverted; see example files)

\texttt{shockssequence}: a sequence of unforeseen shocks under which one
wants to solve the model

\texttt{irfshock}: label for innovation for IRFs, from Dynare mod file (one
or more of the `varexo')

\texttt{nperiods}: simulation horizon (can be longer than the sequence of
shocks defined in \texttt{shockssequence}; in some cases it needs to be long
enough to ensure convergence back to the reference model at the end of the
simulation horizon)

\texttt{curb\_retrench: }a scalar equal to 0 or 1. Default is 0. When set to
0, it updates the guess based on the full information gained at each
iteration. When set to 1, it updates in a manner similar to a Gauss-Jacobi
scheme, slowing the iterations down.

\texttt{maxiter}: maximum number of iterations to solve model until
convergence (20 if not specified)

\texttt{init: }the initial position for the vector of state variables, in
deviation from steady state. (If not specified, the default is steady state)

\medskip
\end{enumerate}

\section{Additional Notes}

\begin{enumerate}
\item Suppose the original occasionally binding constraint in the model is 
\begin{equation*}
i_{t}\geq \log \left( \phi \cdot I_{ss}\right)
\end{equation*}%
where $i_{t}$ is natural logarithm of investment (see mod file), $I_{ss}$ is
steady state investment in levels, and $\phi $ is a parameter.

For the constraint to be violated, in the candidate solution calculated
under the assumption that the constraint does not bind the following must be
true%
\begin{equation*}
i_{t}<\log \left( \phi \cdot I_{ss}\right)
\end{equation*}%
Rewrite each variable as%
\begin{equation*}
x_{t}\equiv \widetilde{x}_{t}+x_{ss}
\end{equation*}

In the runsim\_irrcap.m code, the constraint will have to be expressed in
linearized form as%
\begin{equation*}
\widetilde{i}_{t}+i_{ss}<\log \left( \phi \cdot I_{ss}\right)
\Longleftrightarrow \widetilde{i}_{t}<\log \left( \phi \right)
\end{equation*}

and the string will be 
\begin{equation*}
i<log(PHII)
\end{equation*}

Therefore note that in the mod file, $i$ will denote the variable, whereas
in the \texttt{constraint} the same $i$ will denote the variable minus its
steady state level.

\item In all runsim\_*.m files, we declare M\_ and oo\_ to be global
variables (for use by Dynare)

\item In the toolbox, \texttt{constraint} and \texttt{constraint\_relax}
only admit contemporaneous endogenous variables. Note that this is not
without loss of generality since appropriate redefinitions can accommodate a
general lead and lag structure.

\item One does not need to specify the steady state of any alternative
model, since all models are approximated around the steady state of the
reference model (which needs to satisfy the Blanchard-Kahn conditions)

\item Values for the parameters (whether in the mod or in the steady state
file) are specified only in the reference model. The parameter values
specified in the alternative model are ignored by the code\ (but not the
list of parameters). If a parameter only enters the alternative model, it
needs to be declared as parameter and assigned a value in the reference
model.
\end{enumerate}

\end{document}
