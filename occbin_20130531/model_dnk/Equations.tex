%2multibyte Version: 5.50.0.2890 CodePage: 65001
%% This document created by Scientific Word (R) Version 2.0
%For letter size
%\setlength{\topmargin}{-4mm}


\documentclass[thmsa,letterpaper]{article}
%%%%%%%%%%%%%%%%%%%%%%%%%%%%%%%%%%%%%%%%%%%%%%%%%%%%%%%%%%%%%%%%%%%%%%%%%%%%%%%%%%%%%%%%%%%%%%%%%%%%%%%%%%%%%%%%%%%%%%%%%%%%%%%%%%%%%%%%%%%%%%%%%%%%%%%%%%%%%%%%%%%%%%%%%%%%%%%%%%%%%%%%%%%%%%%%%%%%%%%%%%%%%%%%%%%%%%%%%%%%%%%%%%%%%%%%%%%%%%%%%%%%%%%%%%%%
\usepackage{amsmath}
\usepackage{sw20jart}
\usepackage{harvard}
\usepackage[letterpaper,mag=1000]{geometry}

\setcounter{MaxMatrixCols}{10}
%TCIDATA{TCIstyle=Article/art4.lat,jart,sw20jart}

%TCIDATA{OutputFilter=LATEX.DLL}
%TCIDATA{Version=5.50.0.2890}
%TCIDATA{Codepage=65001}
%TCIDATA{<META NAME="SaveForMode" CONTENT="1">}
%TCIDATA{BibliographyScheme=Manual}
%TCIDATA{LastRevised=Monday, June 03, 2013 11:31:45}
%TCIDATA{<META NAME="GraphicsSave" CONTENT="32">}
%TCIDATA{Language=American English}
%TCIDATA{<META NAME="PrintViewPercent" CONTENT="100">}
%TCIDATA{ComputeDefs=
%$\sigma ^{2}=y^{2}$
%$z_{t}$
%}


\input tcilatex
\geometry{margin=.8in,top=1in,bottom=1in,footskip=24pt}
\renewcommand{\theequation}{\arabic{equation}} 

\begin{document}

\author{Matteo Iacoviello \\
%EndAName
FRB}
\title{Equations for New-keynesian model with government spending}
\date{%
%TCIMACRO{\TeXButton{Today}{\today}}%
%BeginExpansion
\today%
%EndExpansion
}
\maketitle

\section{Model Equations}

We sketch here the equations describing the equilibrium of the model. We
drop the $t$ subscript to denote the steady-state value of a particular
variable.

\begin{eqnarray}
c_{t}+k_{t} &=&w_{t}n_{t}+\left( RR_{t}z_{t}+1-\delta \right) k_{t-1}+\left(
1-\frac{1}{X_{t}}\right) Y_{t}-\tau _{t}Y-b_{t}Y+R_{t-1}b_{t-1}Y  \TCItag{1}
\\
u_{c,t} &=&\beta E_{t}\left( \frac{u_{c,t+1}R_{t}}{\pi _{t+1}}\right)  
\TCItag{2} \\
u_{c,t}v_{t} &=&\beta E_{t}\left( u_{c,t+1}\left( RR_{t+1}+\left( 1-\delta
\right) v_{t+1}\right) \right)   \TCItag{3} \\
u_{c,t}w_{t} &=&u_{n,t}X_{wt}  \TCItag{4} \\
Y_{t} &=&A_{t}n_{t}^{1-\mu }\left( z_{t}k_{t-1}\right) ^{\mu }  \TCItag{5} \\
\left( 1-\mu \right) Y_{t} &=&X_{pt}w_{t}n_{t}  \TCItag{6} \\
\mu Y_{t} &=&X_{pt}RR_{t}z_{t}k_{t-1}  \TCItag{7} \\
\ln \pi _{t}-\iota _{\pi }\ln \pi _{t-1} &=&\beta \left( E_{t}\ln \pi
_{t+1}-\iota _{\pi }\ln \pi _{t}\right) -\varepsilon _{\pi }\ln \left(
X_{pt}/X_{p}\right) +u_{p,t}  \TCItag{8} \\
\omega _{c,t}-\iota _{wc}\ln \pi _{t-1} &=&\beta \left( E_{t}\omega
_{c,t+1}-\iota _{wc}\ln \pi _{t}\right) -\varepsilon _{wc}\ln \left(
X_{wt}/X_{wc}\right)   \TCItag{9} \\
R_{t} &=&\left( R_{t-1}\right) ^{r_{R}}\pi _{t}^{r_{\pi }\left(
1-r_{R}\right) }\left( \frac{GDP_{t}}{GDP_{t-1}}\right) ^{r_{Y}\left(
1-r_{R}\right) }\overline{rr}^{1-r_{R}}  \TCItag{10}
\end{eqnarray}

Additional definitions given by%
\begin{eqnarray}
u_{c,t} &=&\frac{1-\varepsilon _{c}}{1-\beta \varepsilon _{c}}\left( \frac{1%
}{c_{t}-\varepsilon _{c}c_{t-1}}-\frac{\beta \varepsilon _{c}}{%
c_{t+1}-\varepsilon _{c}c_{t}}\right)   \TCItag{11} \\
u_{n,t} &=&\tau n_{t}^{\eta }  \TCItag{12}
\end{eqnarray}

The equation for capacity is%
\begin{equation}
RR_{t}=\left( \frac{1}{\beta }-\left( 1-\delta \right) \right) \left( \frac{%
\zeta }{1-\zeta }z_{t}+1-\frac{\zeta }{1-\zeta }\right)   \tag{13}
\end{equation}

and the definition of investment is%
\begin{equation}
i_{t}=k_{t}-\left( 1-\delta \right) k_{t-1}  \tag{14}
\end{equation}

The optimality conditions related to investment are

\begin{equation}
u_{c,t}v_{t}\left( 1-mac_{t}\right) =u_{ct}-\beta G_{C}u_{c,t+1}mac_{t+1} 
\tag{15}
\end{equation}%
where:%
\begin{equation}
mac_{t}=\frac{d\phi _{t}}{di_{t}}=\phi \left( i_{t}-i_{t-1}\right)   \tag{16}
\end{equation}

The remaining equations describe stochastic processes for the shocks and the
evolution of government spending and the government budget constraint%
\begin{eqnarray}
g_{t}Y &=&\left( 1-\rho _{g}\right) gY+\rho _{g}g_{t-1}Y+\varepsilon _{gt}Y 
\TCItag{17} \\
g_{t}Y &=&R_{t-1}b_{t-1}Y=\tau _{t}Y+b_{t}Y  \TCItag{18} \\
\tau _{t}Y &=&\rho _{\tau }\tau _{t-1}Y+\left( 1-\rho _{\tau }\right) \left(
\varepsilon _{\tau b}b_{t-1}Y+\varepsilon _{\tau g}g_{t}Y\right)  
\TCItag{19}
\end{eqnarray}

\end{document}
